\documentclass{article}
\usepackage[brazil]{babel}
\usepackage[utf8]{inputenc}
\usepackage{indentfirst}
\usepackage[T1]{fontenc}
\usepackage[a4paper, left=20mm, right=20mm, top=20mm, bottom=20mm]{geometry}
\usepackage[colorlinks, urlcolor=blue, citecolor=red]{hyperref}
\usepackage{tikz, amsmath}
\usetikzlibrary{positioning}


\begin{document}

\title{\textbf{Redes Bayesianas}}
\author{
    Lucas Ribeiro Neis\\
     {\texttt{lucasrneis@gmail.com}}
     \and
     Vinícius Couto Biermann\\
      {\texttt{viniciusbiermann@hotmail.com}}
      \vspace{-50mm}
}
\date{Outubro 2016}

\maketitle

\section{Introdução}

Foi proposto o tema de explicar avistamentos recentes de zumbis no campus da UFSC para que seja posto em prática o conhecimento em 
raciocínio probabilístico e redes Bayesianas. Foram dadas as seguintes variáveis aleatórias:

\begin{itemize}
    \item Mg = Um feitiço mágico foi lançado \{V, F\}
    \item Vr = Há um surto viral de H1Z1 \{V, F\}
    \item Hp = Você vê um Hipogrifo (uma criatura mágica) \{V, F\}
    \item Zb = Você vê um Zumbi \{V, F\}
    \item Tp = Você vê Viajantes do Tempo em ternos estranhos tentando conter o surto viral. \{V, F\}
\end{itemize}

Foi também fornecido uma lista de fatos e probabilidades conjuntas representados pela seguinte rede Bayesiana:

\begin{center}
\begin{tikzpicture}
    \begin{scope}[every node/.style={circle, draw, align=center}]
        \node (Hp) at (0, 0)    {$Hp$};
        \node (Mg) at (1, 1.5)  {$Mg$};
        \node (Zb) at (2, 0)    {$Zb$};
        \node (Vr) at (3, 1.5)  {$Vr$};
        \node (Tp) at (4, 0)    {$Tp$};
    \end{scope}

    \begin{scope}[>=latex, thick]
        \draw[->] (Mg) -- (Hp);
        \draw[->] (Mg) -- (Zb);
        \draw[->] (Vr) -- (Zb);
        \draw[->] (Vr) -- (Tp);
    \end{scope}

    \begin{scope}[every node/.style={node distance=2mm}]
        \node[above left = of Mg] {
            \begin{tabular}{c|c}
                $Mg$ & 0,1      \\  \hline
                $\neg Mg$ & 0.9  \\
            \end{tabular}
        };

        \node[above right = of Vr] {
            \begin{tabular}{c|c}
                $Vr$ & 0,2      \\  \hline
                $\neg Vr$  & 0.8 \\
            \end{tabular}
        };

        \node[left = of Hp] {
            \begin{tabular}{c|c|c}
                          & $Hp$ & $\neg Hp$  \\ \hline
                $Mg$      & 0.8      & 0.2     \\ \hline
                $\neg Mg$ & 0.7      & 0.3      \\
            \end{tabular}
        };

        \node[right = of Tp] {
            \begin{tabular}{c|c|c}
                          & $Tp$ & $\neg Tp$  \\ \hline
                $Vr$      & 0.3      & 0.7     \\ \hline
                $\neg Vr$ & 0.1      & 0.9      \\
            \end{tabular}
        };

        \node[below = 2mm of Zb] {
            \begin{tabular}{c|c|c|c}
                          &           & $Zb$ & $\neg Zb$  \\ \hline
                $Mg$      & $Vr$      & 0.6      & 0.4     \\ \hline
                $Mg$      & $\neg Vr$ & 0.5      & 0.5      \\ \hline
                $\neg Mg$ & $Vr$      & 0.4      & 0.6       \\ \hline
                $\neg Mg$ & $\neg Vr$ & 0.01     & 0.99       \\
            \end{tabular}
        };
    \end{scope}
\end{tikzpicture}
\end{center}

\section{Questões}

\begin{enumerate}
    \item Qual é a probabilidade de não ver um Zumbi dado que um feitiço mágico não foi lançado e não há surto viral? \\
        Conseguimos os valores diretamente da tabela de probabilidades de avistamentos de um Zumbi.
        \begin{align*}
            P(\neg Zb \mid \neg Mg \land \neg Vr) = 0,99 = 99\%
        \end{align*}
    
    \item Qual a probabilidade do mundo estar no seguinte estado: $(Mg, Hp, Zb, Vr, Tp)$?
    
        \begin{align*}
            P(Mg, Hp, Zb, Vr, Tp)&=P(Mg)*P(Vr)*P(Hp \mid Mg)*P(Tp \mid Vr)*P(Zb \mid Mg \land Vr) \\
                                 &= 0,1*0,2*0,8*0,3*0,6 = 0,00288 \approx 0,288\%
        \end{align*}
    
    \item Qual a probabilidade de ser ver um Zumbi?
        \begin{align*}
            P(Zb) =\ &P(Zb \mid Mg \land Vr)*P(Mg)*P(Vr)\ + \\
                    &P(Zb \mid Mg \land \neg Vr)*P(Mg)*P(\neg Vr)\ + \\
                    &P(Zb \mid \neg Mg \land Vr)*P(\neg Mg)*P(Vr)\ + \\
                    &P(Zb \mid \neg Mg \land \neg Vr)*P(\neg Mg)*P(\neg Vr) \\
                    =\ &0,6 * 0,1 * 0,2 + 0,5 * 0,1 * 0,8 + 0,4 * 0,9 * 0,2 + 0,01 * 0,9 * 0,8 \\
                    =\ &0,012 + 0,04 + 0,072 + 0,0072 = 0,1312 \approx 13,1\%
        \end{align*}
    
    \item Qual a probabilidade de ver um Zumbi dado que está havendo um surto viral?
        \begin{align*}
            P(Zb|Vr) &= P(Zb|Vr \land Mg)*P(Mg)+P(Zb|Vr \land \neg Mg)*P(\neg Mg) \\
                     &= 0,6 * 0,1 + 0,4 * 0,9 = 0,06 + 0,36 = 0,42 = 42\%
        \end{align*}
    
    \item Qual a probabilidade de ser ver um Hipogrifo dado que você conseguiu ver um Zumbi?
        \begin{align*}
            P(Hp \mid Zb) &= \frac{P(Hp, Mg, Zb)+P(Hp, \neg Mg, Zb)}{P(Zb)} \\
                          &= \frac{P(Mg)*P(Hp \mid Mg)*P(Zb \mid Mg)+P(\neg Mg)*P(Hp \mid \neg Mg)*P(Zb \mid \neg Mg)}{P(Zb)} \\
                          &= \frac{0,1*0,8*P(Zb \mid Mg)+0,9*0,7*P(Zb \mid \neg Mg)}{0,1312}
        \end{align*}
        
        Para continuar precisamos saber as probabilidades de $P(Zb \mid Mg)$ e $P(Zb \mid \neg Mg)$.
        
        \begin{align*}
            P(Zb \mid Mg) &= P(Zb \mid Mg \land Vr)*P(Vr)+P(Zb \mid Mg \land \neg Vr)*P(\neg Vr) \\
                          &= 0,6 * 0,2 + 0,5 * 0,8 = 0,12 + 0,4 = 0,52 \\ 
                          \\
            P(Zb \mid \neg Mg) &= P(Zb \mid \neg Mg \land Vr)*P(Vr)+P(Zb \mid \neg Mg \land \neg Vr)*P(\neg Vr) \\
                               &= 0,4 * 0,2 + 0,01 * 0,8 = 0,08 + 0,008 = 0,088
        \end{align*}
        
        Podemos agora continuar na conta anterior substituindo os valores:
        
        \begin{align*}
            P(Hp \mid Zb) &= \frac{0,1*0,8*0,52+0,9*0,7*0,088}{0,1312} \\
                          &= \frac{0,0416+0,05544}{0,1312} \\
                          &= \frac{0,09704}{0,1312} = 0,7396 \approx 74\%
        \end{align*}
    
    \item Qual a probabilidade de se ver um Zumbi dado que você conseguiu ver um Hipogrifo?
        \begin{align*}
            P(Zb \mid Hp) &= \frac{(P(Zb, Mg, Vr, Hp) + P(Zb, \neg Mg, Vr, Hp) + P(Zb, Mg, \neg Vr, Hp) + P(Zb, \neg Mg, \neg Vr, Hp))}{ P(Hp)} \\         
		                  &= \frac{(P(Mg) * P(Vr) * P(Hp \mid Mg) * P(Zb \mid Mg \land Vr)}{P(Hp)} \\ 
		                  &+ \frac{(P(\neg Mg) * P(Vr) * P(Hp \mid \neg Mg) * P(Zb \mid \neg Mg \land Vr)}{P(Hp)} \\
		                  &+ \frac{(P(Mg) * P(\neg Vr) * P(Hp \mid Mg) * P(Zb \mid Mg \land \neg Vr)}{P(Hp)} \\
		                  &+ \frac{(P(\neg Mg) * P(\neg Vr) * P(Hp \mid \neg Mg) * P(Zb \mid \neg Mg \land \neg Vr)}{P(Hp)} \\
		                  &= \frac{0,1 * 0,2 * 0,8 * 0,6 + 0,9 * 0,2 * 0,7 * 0,4 + 0,1 * 0,8 * 0,8 * 0,5 + 0,9 * 0,8 * 0,7 * 0,01}{P(Hp)}
        \end{align*}

        Agora precisamos saber qual a probabilidade de vermos um hipogrifo.

        \begin{align*}
            P(Hp) &= P(Hp \mid Mg) * P(Mg) + P(Hp \mid \neg Mg) * P(\neg Mg) \\
                  &= 0,8 * 0,1 + 0,7 * 0,9 = 0,71\\
        \end{align*}
        Assim, podemos concluir:
        
        \begin{align*}
            P(Zb \mid Hp) &= \frac{0,1 * 0,2 * 0,8 * 0,6 + 0,9 * 0,2 * 0,7 * 0,4 + 0,1 * 0,8 * 0,8 * 0,5 + 0,9 * 0,8 * 0,7 * 0,01}{0,71}\\
                       &= \frac{0,0096 + 0,0504 + 0,032 + 0,00504}{0,71}\\
                       &= \frac{0,9704}{0,71}
                        = 0,136676056 \approx 13,67\% \approx 13,7\%
        \end{align*}
        
    
    \item Qual a probabilidade de se ver um Zumbi dado que você conseguiu ver um Hipogrifo e um Viajante do Tempo?
    
        \begin{align*}
            P(Zb \mid Hp \land Tp) = &(P(Zb,Hp,Tp,Mg,Vr)+P(Zb,Hp,Tp,Mg,\neg Vr)\ + \\
	                                 &P(Zb,Hp,Tp,\neg Mg,Vr)+P(Zb,Hp,Tp,\neg Mg,\neg Vr))\ /\ P(Hp \land Tp) \\
	                                 \\
	                               = &(P(Mg)*P(Vr)*P(Hp \mid Mg)*P(Tp \mid Vr)*P(Zb \mid Mg \land Vr)\ + \\
	                                 &P(Mg)*P(\neg Vr)*P(Hp \mid Mg)*P(Tp \mid \neg Vr)*P(Zb \mid Mg \land \neg Vr)\ + \\
	                                 &P(\neg Mg)*P(Vr)*P(Hp \mid \neg Mg)*P(Tp \mid Vr)*P(Zb \mid \neg Mg \land Vr)\ + \\
	                                 &P(\neg Mg)*P(\neg Vr)*P(Hp \mid \neg Mg)*P(Tp \mid \neg Vr)*P(Zb \mid \neg Mg \land \neg Vr))\ /\ P(Hp \land Tp) \\
	                                 \\
	                               = &(0,1 * 0,2 * 0,8 * 0,3 * 0,6 + \\
	                                 &0,1 * 0,8 * 0,8 * 0,1 * 0,5 + \\
	                                 &0,9 * 0,2 * 0,7 * 0,3 * 0,4 + \\
	                                 &0,9 * 0,8 * 0,7 * 0,1 * 0,01)\ /\ (P(Hp)*P(Tp)
        \end{align*}
        
        Da questão anterior conseguimos o valor de $P(Hp)$, mas ainda temos que calcular o valor de $P(Tp)$ para finalizar o cálculo.
        \begin{align*}
            P(Tp) &= P(Tp \mid Vr) * P(Vr) + P(Tp \mid \neg Vr) * P(\neg Vr) \\
                  &= 0,3 * 0,2 + 0,1 * 0,8  = 0,06 + 0,08 = 0,14
        \end{align*}
        Agora podemos terminar o cálculo anterior.
        \begin{align*}
            P(Zb \mid Hp \land Tp) &= \frac{0,00288+0,0032+0,01512+0,000504}{0,71*0,14}\\
                                   &= \frac{0,021704}{0,0994} = 0,21835010 \approx 21,8\%
        \end{align*}
    
\end{enumerate}

\end{document}
